
\def \Subject {گزارش پیاده سازی مدل }
\def \Course {درس یادگیری ماشین}
\def \Author {کسرا سینایی، امیرحسین افخمی}
\def \Report {گزارش نهایی}
\def \StudentNumber {۸۱۰۶۹۶۲۵۴، 810696206}

\begin{center}
\vspace{.4cm}
{\bf {\huge \Subject}}\\
{\bf \Large \Course}
\vspace{.2cm}
\end{center}
{\bf \Author }  \\
{\bf شماره دانشجویی:\ \StudentNumber}
\hspace{\fill} 
{\Large \Report} \\
\hrule
\vspace{0.8cm}

\clearpage

%\huge{\Subject}\\[1.5 cm]
%\chapterauthor{\Author~ : \StudentNumber}

\section{فیچر اکسترکشن}


\section{پیش پردازش}


\section{طبقه بندی}

\section{خوشه بندی}







\section{نمایش داده ها در ابعاد پایین تر} 

\subsection{موسیقی ترک‌های ایران}

        
        \begin{table}[!h]
\centering
\begin{tabular}{|c|c|c|}
	\hline
    ردیف &  شکل موسیقی رایج &  نمونه\\
	\hline\hline
    ۱ & حماسه و رزم & شاهسونی \\
	\hline
    ۲ & روایت و نقل& کوراوغلی-غریب و شاه‌صنم \\
	\hline
	۳ & کار و تلاش & یئل‌یئل‌های کشاورزی
\\	\hline
	۴ &آئین‌های شادی  & آیاقها- سوبولاغی \\
	\hline
	۵ & آئین‌های سوگ و عزا & اوخشاما-نوحه \\
	\hline
	۶ & اوقات خاص & گوزللمه-اکبری \\
	\hline
	۷ & اندیشه و طریقت & عرفانی-دیوانی-قلندری  \\
	\hline
	۸ & آئین‌های باورمدارانه & سویدقازان-کوسه‌گلدی-ناقلدی\\
	\hline
	۹ &  شوخ و شنگ & ننه آغلاتما-رستم‌بازی\\
	\hline
	۱۰ & نمایش‌ها و خرده‌نمایش‌های آئینی & دیلجالمانی-تعزیه-شاخ‌سین  \\
	\hline
	۱۱ & توصیفی و ستایشی  & ائل کوچدو-ائل هاواسی  \\
	\hline
\end{tabular}
\caption{اشکال موسیقی رایج در جامعه‌ی ترک‌های ایران}
\label{t1}
\end{table}


\begin{thebibliography}{99}
	\bibitem{ref} آشنایی با موسیقی نواحی ایران . هوشنگ جاوید. انتشارات سوره مهر. چاپ دوم. سال ۹۳
	\resetlatinfont
	\bibitem{tzanetakis2002musical}Tzanetakis, G. \& Cook, P. Musical genre classification of audio signals. {\em IEEE Transactions On Speech And Audio Processing}. \textbf{10}, 293-302 (2002)
\end{thebibliography}