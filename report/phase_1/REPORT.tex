
\def \Subject {گزارش جمع آوری داده }
\def \Course {درس یادگیری ماشین}
\def \Author {کسرا سینایی، امیرحسین افخمی، پارسا شفیعی و رها ارباسی }
\def \Report {گزارش فاز اول}
\def \StudentNumber {۸۱۰۶۹۶۲۵۴، 810696206، و ۸۱۰۶۹۹۰۳۵}

\begin{center}
\vspace{.4cm}
{\bf {\huge \Subject}}\\
{\bf \Large \Course}
\vspace{.2cm}
\end{center}
{\bf \Author }  \\
{\bf شماره دانشجویی:\ \StudentNumber}
\hspace{\fill} 
{\Large \Report} \\
\hrule
\vspace{0.8cm}

\clearpage

%\huge{\Subject}\\[1.5 cm]
%\chapterauthor{\Author~ : \StudentNumber}

\section{جمع آوری داده}
در مرحله اول تمامی مسائل یادگیری ماشین، ابتدا نیاز است تا داده مورد نیاز و متناسب با مسئله‌ای که قصد حل کردن آن را داریم جمع آوری شود.
در این پروژه هدف اصلی، کلاس‌بندی آهنگ‌های محلی با پنج گویش مختلف است و لذا در اولین مرحله از پروژه به جمع آوری آهنگ‌های متناسب با این هدف پرداخته شد.
برای تحقق این امر، هر یک از 4 اعضای گروه به طور جداگانه، حداقل 10 آهنگ برای هر یک از گویش‌های تعیین شده پیدا و در گوگل درایو آپلود نمود.
آهنگ‌های جمع آوری شده شامل پنج گویش مختلف لری، کردی، ترکی، آذری و بندری بودند.
بدیهی است که هر یک از آهنگ‌های مربوط به هر یک از گویش‌های فوق دارای خصوصیات خاص خود هستند اما در بسیاری از موارد شخصی که هیچگونه دانش اولیه‌ای درباره این گویش‌ها نداشته باشد
نمی‌تواند آن‌ها را از هم تشخیص دهد. این مورد در خصوص بعضی از گویش‌ها بسیار مشکل سازتر است. در ادامه به بررسی تفاوت‌ها و همچنین چالش‌ها و سختی‌های پیش رو در این پروژه پرداخته می‌شود.

\section{چالش‌ها و سختی‌ها}
اولین مشکلی که شاید به ذهن متبادر شود این است که به علت قدیمی بودن بعضی از آهنگ‌های جمع آوری شده، کیفیت آن‌ها پایین و دارای
نویز باشد که وجود این نویز می‌تواند در طبقه بندی و خوشه بندی داده‌ها مشکل‌ساز باشد. \\
نکته بعدی آن است که برای طبقه بندی و خوشه بندی داده‌ها نیاز به
extraction feature
داریم و از آن جایی 
که داده‌های ما آهنگ هستند، احتیاج است تا به حوزه سیگنال برده شوند و به همین دلیل برای داشتن فیچرهای مناسب جهت طبقه بندی 
نیاز به دانستن دانش حوزه پردازش سیگنال می‌باشد. همچنین برای پیش پردازش داده‌ها نیز دانستن این دانش ضروری است. \\
نکته بسیار مهم دیگر نیز آن است که تم آهنگ موجود در بعضی از آهنگ‌های محلی با گویش‌های متفاوت بسیار به هم شبیه است که 
این موضوع طبقه بندی آن‌ها را دشوار می‌کند، به طور مثال در آهنگ‌های کردی و لری که تم غمگین دارند، تشخیص و کلاس‌بندی آن‌ها
از هم دشوار است. \\
همچنین این تفاوت موجود در تم‌های آهنگ‌های محلی می‌تواند طبقه بند را به اشتباه بندازد، به طور مثال طبقه بند و خوشه بند داده‌ها را از نظر شاد و 
غمگین بودن کلاس‌بندی و خوشه بندی کنند. \\

\section{مروری بر موسیقی نواحی ایران}

\subsection{موسیقی ترک‌های ایران}
    موسیقی ترک‌های ایران گستره‌ی پیوسته‌ای از منطقه‌ی آذربایجان شرقی ,اردبیل, هشتپر و توالش, قزوین, زنجان, همدان, استان مرکزی, فارس, , گلستان , خراسان , آذربایجان غربی و جبهه‌ای از خاک مازندران را در بر دارد.
    \subsubsection{نغمات در موسیقی ترک‌های ایران}
        راست شور سه‌گاه چهارگاه بیات شیراز (زند) شوشتری همایون  ماهور هندی بیات قاجار سه‌گاه میرزا حسین  سه‌گاه زابل  سه‌گاه خارج و ماهور.\\
        
        این دستگاه‌ها بر هفت لاد استوار شده اند :\\
        
        راست- شور - سه‌گاه-شوشتری-بیات شیراز- چهارگاه-همایون
        که هرکدام سه قسمت مجزا دارند:\\
        
        رنگ(درآمد)+ برداشت+مایه
        
        \begin{table}[!h]
\centering
\begin{tabular}{|c|c|c|}
	\hline
    ردیف &  شکل موسیقی رایج &  نمونه\\
	\hline\hline
    ۱ & حماسه و رزم & شاهسونی \\
	\hline
    ۲ & روایت و نقل& کوراوغلی-غریب و شاه‌صنم \\
	\hline
	۳ & کار و تلاش & یئل‌یئل‌های کشاورزی
\\	\hline
	۴ &آئین‌های شادی  & آیاقها- سوبولاغی \\
	\hline
	۵ & آئین‌های سوگ و عزا & اوخشاما-نوحه \\
	\hline
	۶ & اوقات خاص & گوزللمه-اکبری \\
	\hline
	۷ & اندیشه و طریقت & عرفانی-دیوانی-قلندری  \\
	\hline
	۸ & آئین‌های باورمدارانه & سویدقازان-کوسه‌گلدی-ناقلدی\\
	\hline
	۹ &  شوخ و شنگ & ننه آغلاتما-رستم‌بازی\\
	\hline
	۱۰ & نمایش‌ها و خرده‌نمایش‌های آئینی & دیلجالمانی-تعزیه-شاخ‌سین  \\
	\hline
	۱۱ & توصیفی و ستایشی  & ائل کوچدو-ائل هاواسی  \\
	\hline
\end{tabular}
\caption{اشکال موسیقی رایج در جامعه‌ی ترک‌های ایران}
\label{t1}
\end{table}
سازها-آلات و ادوات موسیقی ترک‌های ایران:\\
۱.قوپوز-گارمون(آکاردئون روسی)-سبخ\\
۲.قاوال(دایره)-قره‌نی(کلارینت)-سوتک\\
۳.بالابان-نی-زل(زبان)
۴.داوارجیغ-نی‌لبک-تارترکی\\
۵.جالمان(سرنا)-کمانچه\\
۶.ناقارا(طبل دستی)-دهل(طبل عزا)
\subsection{موسیقی کردهای ایران}
\subsubsection{نغمات موسیقی کردهای ایران}
%\subsubsubsection{سیاوچمانه}
این دسته آواز پانزده طرز نغمگی دارد:\\
بادباده-عروسی-شیون-مور-سحری-ده‌ره‌ی-بزمی شیخانه-دانه‌بیره-لالایه-آواز گ-مانگه‌گیران-بزمی‌چیلا-بزمی‌گوش- و شکاری
\\هوره:\\
این مقام دارای شوگی غریب است و نغمات آن به نام‌های زیر شناخته می‌شوند:\\
بای‌بندای-بنیریچر-دودنگی--باریه-غریبی-ساروخانی- گل و دره-پاموری-هجرانی--مجنونی-هی‌لاوه- سواره‌چر-طرز رستم-گوله خاک- بالادستانی-جلوشاهی- و ...\\
حیران:\\
گورانی:\\
سازهای موسیقی کردی:\\
دف-دهل-دوطبله-تاس-شمشال(نی‌فلزی)- دوزله-نرمه‌نای-پره‌تاس-بلور-دیوان-کمانچه-تنبور-سرنا-زیل(سنج)
\subsection{موسیقی لرهای ایران}
 به هفت بخش تقسیم می‌شود که عبارتند از موسیقی و ترانه‌های غنایی و عاشقانه، موسیقی و ترانه‌های حماسه رزمی، موسیقی و ترانه‌های سوگواری، موسیقی و ترانه‌های فصول، موسیقی و ترانه‌های کار، موسیقی و ترانه‌های طنز و سرودهای مذهبی.

موسیقی و ترانه‌های غنایی و عاشقانه: شامل ترانه‌ها و آهنگ‌هایی است در وصال یا فراق معشوق مانند ترانه‌های هی لو، بینا بینا، کیودار یا نغمه‌های شیرین و خسرو، ساری خوانی، میربگی (میرونه) و ده‌ها ترانه دیگر که در مقام‌های مختلف موسیقی لری اجرا می‌گردد.\\

موسیقی و ترانه‌های حماسه رزمی: این سرودها بیانگر ارزش‌های حماسی و رزمی جنگاوران در میان ایل هستند. مانند جنگ لرو، دایه دایه یا مقام‌های موسیقی بدون کلام که در رزم‌گاه و مسابقه به کار رفته‌اند، مانند جنگه را، سوارهو و نقاره.\\

موسیقی و ترانه‌های سوگواری: این موسیقی بیشتر جنبه آیینی داشته‌است و در مواقع سوگواری از روزگار کهن تاکنون کاربرد فراوانی دارد، مانند چمری یا سایر مقام‌ها، از جمله، سحری، پاکتلی، شیونی و ده‌ها مقام دیگر.\\

موسیقی و ترانه‌های فصول: موسیقی و ترانه‌های ویژه فصول مختلف مانند برزه کوهی، ماله ژیری، کوچ بارو.
موسیقی و ترانه‌های کار: به منظور سهولت و تسریع در کار مردان و زنان ایلاتی، این ترانه‌ها به صورت فردی یا دسته جمعی خوانده می‌شود، مانند ترانه‌های گل‌درو (برزیگری)، هوله (خرمن کوبی)، مشک زنی، شیردوشی و چوپانی.\\

موسیقی و ترانه‌های طنز: این ترانه‌ها اغلب به صورت فی البداهه در هجو شخص یا موضوع یا مکانی سروده شده و برخی اوقات نیز با حرکات نمایشی طنز آلود فرد یا افرادی همراه بوده‌است.
\\
سرودهای مذهبی: بر اساس کلام‌های یارسان (از سروده‌های اهل حق) بوده، جنبه عرفانی و اعتقادی آن بسیار عمیق است. مانند ضامن آهو، سرای خاموشان (شهر بی‌صدا) و دوازده کلام یاری.\\


سازهای موسیقی لری:\\

از سازهای اصلی رایج می‌توان به سرنا-تنبور-کمانچه و دهل اشاره کرد همچنین می توان به دوزله -تنبک فلزی- تنبک چوبی-نی- دایره-نی‌لبک-دوطبله سفالی-بوق شاخی و برنجی- زنجیر عزا-فلوت- سنج و طبل عزا- زنگ و ضرب زورخانه-

\subsection{موسیقی گیلان و تالش}
%\subsubsubsection{نغمات موسیقی گیلان و تالش}
مهم‌ترین آوازهای گیلانی در بخش ترانه به دسته‌های زیر تقسیم‌بندی می‌شود:
۱.ترانه‌های کار و تلاش: مانند ترانه‌های شالیزار-مزرعه‌ی چای و میوه چینی\\
۲.ترانه‌های سور: که به صورت چندصدایی و جمعی خوانده می‌شود\\
۳.ترانه‌های حزن و فراق\\
از مهم‌ترین ترانه‌ها و تصنیف‌های گالش در گیلان می‌توان به موارد زیر اشاره کرد:\\
لاکودونه-عزب-لاکو-هیبه-هالی‌آوه- زرنگیس- جوان دیلمانی یاغی وابوی-رعنا- - شورته‌ی ماشل‌زه- می خون تی مارگردن-لی لی جونه- جون دوستی یی-دراز ترمونه‌ی جون- وای خدا خورده تاب- اشاره نمود
 سازهای موسیقی گیلان:\\
 لله(نی هفت‌بند محلی)- ساز یا زورنا(همان سرنا)--دایره-جفت ناقاره-تکه ناقاره- کرنای کدویی- کرنای شاخی-لبک-دف- دایره زنگی-کمانچه- تنبوره‌ی تالش- شیپور مسی- تشت مسی- پیاله زنگی (سنجک)- سوت سفالی- کرب- زنجیر.\\
 \subsection{موسیقی بندری}
 %\subsubsubsection{نغمات موسیقی جنوب ایران}
 بندری : خود یکی از مقام‌های موسیقی جنوب ایران است از محبوب‌ترین شیوه های اجرای موسیقی در جنوب است که بسیار شبیه موسیقی آفریقایی و متأثر از موسیقی مهاجران آفریقایی به جنوب ایران است. موزیکی شاد و ریتم‌دار است. این موسیقی با استفاده از ساز‌های کوبه‌ای تومبا، تمپو و نی انبان که یک ساز بادی است و در مر
 اسم‌های شاد و عروسی‌ها نواخته می‌شود.

 \section{مقایسه آهنگ‌های محلی}
در این قسمت مخصرا به تفاوت‌هایی که در حین جمع آوری داده با آن‌ها مواجه شدیم اشاره می‌کنیم. (بدون دید یادگیری ماشین)

\subsection{گویش و زبان}
بدیهی‌ترین و قابل تشخیص‌ترین تفاوت بین آهنگ‌های محلی زبان و لهجه خواننده آن‌ها می‌باشد. حتی می‌توان آن را متمایز کننده‌ترین ویژگی در آهنگ‌ها که به ما توانایی تشخیص آهنگ‌های محلی از یکدیگر را بدهد عنوان کرد. سبک، ریتم و وزن‌های بعضا یکسان در آهنگ‌های محلی از نواحی مختلف به گوش می‌خورد. برای مثال اهنگ‌های شاد با وزن ۶/۸ ممکن است در تمام دسته‌بندی‌های مسیله مانند بندری، کردی و گیلکی موجود باشند اما در هیچکدام از آهنگ‌های کردی خواننده گویش گیلکی و ندارد.
\subsection{سازها}
علاوه بر گویش و زبان خواننده برخی سازهای محلی و سنتی متعلق به یک ناحیه خاص بوده و توسط خوانندگان و موزیسین‌های ناحیه‌ای خاص به طور انحصاری استفاده می‌شوند. برای مثال می‌توان به سازهای کوبه‌ای گوناگون در آهنگ‌های بندری اشاره کرد. البته لازم به ذکر است که هم‌پوشانی بین سازهای استفاده شده در گروه‌های مختلف ممکن است به چشم بخورد. همچنین بسیاری از آهنگ‌ها با وجود استایل و سبک محلی و سنتی از سازهای غیرمحلی و عام نیز استفاده کرده‌اند.
\subsection{محتوی و سبک کلی}
ریشه برخی از ترانه‌ها و اشعار استفاده شده در ترانه‌های نواحی مختلف به فرهنگ‌ها، افسانه‌ها و آداب و رسوم مردم آن محله باز می‌گردد. مسلماً محتوی اشعار هر آهنگ در کنار زبان خواننده و اشعار یکی از وجه تمایزهای قابل تشخیص برای یک کاربر ساده (غیر هنرمند) می‌تواند باشد.\\
علاوه بر محتوی سبک غالب آهنگ‌ها نیز نیز در حین جمع آوری داده مورد توجه بوده‌است. برای مثال آهنگ‌های بندری بیشتر آهنگ‌هایی شاد بوده‌اند (نه همه‌ی آن‌ها بلکه بیشتر آن‌ها) و در مقابل اهنگ‌های با وزن حماسی تر و جدی تر در بین ترانه‌های ترکی و لری بیشتر از بندری به گوش می‌خوردند.
\subsection{تکنیک‌های نوازندگی}
موسیقی‌های ایرانی در هفت دستگاه دسته بندی می‌شوند. غالبا ترانه‌های محلی و دستگاه‌های آوازی معروف‌تر و پر استفاده تر از زیر مجموعه‌های (گوشه) دستگاه شور می‌باشند. (ابوعطا، دشتی، بیات، افشاری، بیات کرد) با وجود شباهت و یکسان بودن تکنیک‌های آوازی در بین ترانه‌های محلی همچنان بسیاری از موسیقی‌ها از این دسته بندی به طور سفت و سخت پیروی نمی‌کنند. چه بسا برخی آهنگ‌های جمع آوری شده اصلا در دسته بندی‌های موسیقی سنتی ایرانی جای نگیرند.
\subsection{جمع بندی مقایسه ترانه‌ها}
تمام موارد اشاره شده در این بخش فیچرهای قابل تشخیص با گوش انسان هستند و مدل سازی ریاضی این موارد بعضا غیرممکن یا دشوار است. برای اجرای الگوریتم‌های طبقه بندی یا خوشه بندی طبیعتاً از فیچرهای قابل اندازه گیری و قابل مدل سازی مانند موارد زیر استفاده می‌شود:
\begin{description}
    \item[$\bullet$] \lr{Zero-crossing rate}
    \item[$\bullet$] \lr{Spectral Centroid}
    \item[$\bullet$] \lr{Spectral Rolloff}
    \item[$\bullet$] \lr{Mel-Frequency Cepstral Coefficients}
    \item[$\bullet$] \lr{Chroma Frequencies} 
\end{description}
از آن‌جا که هدف این گزارش بررسی مدل‌سازی و استخراج فیچر از فایل‌های صوتی نمی‌باشد این موضوعات بیشتر از این باز نمی‌شوند و هدف از بیان آن صرفا مقایسه ادراک ما از آهنگ و مدل‌سازی آن‌ها بوده است. 